\documentclass{article}
\usepackage[utf8]{inputenc}

\title{CMPS 185 ASGN 4}
\author{tanees}
\date{February 2018}

\begin{document}

\maketitle

\section{Part 1}
\subsection{Proceedings of the ACM on Human-Computer Interaction}
The name of the journal is The Proceedings of the ACM on Human Computer Interaction(HCI). The URL of the journal is https://dl.acm.org/pub.cfm?id=J1598. The journals can be read for free through the UC Santa Cruz internet connection. The publisher is the Association for Computing Machinery. The Editor-in-Chief is Cliff Lampe from the University of Michigan and there are 7 members in the editorial board including the Editors, Information Director and Advisory Board. The maximum length of the paper is 10 pages and the journal does have a double bind review policy.One interesting fact I learned about the journal is that there is a Master Template one must follow in order to submit to ACM. The deadlines are also available to view and this helps which include draft deadlines and dates to expect to receive approval.


\subsection{Artificial Intelligence and Law}

The name of the journal is Artificial Intelligence and Law and can be accessed through the Springer website and the URL is http://www.springer.com/us/computer-science/all-journals-in-computer-science. Springer Nature is the publisher and they provide many journals over many topics. The name of the editors in chief are K.D. Ashely, T. Bench-Capon, G.Sartor. The Editorial Board consists of 28 members. This article does have the Open Access option which means that their work can be published to the world and their copyright remains with them. Their is a fee of 3,000 dollars which is necessary for this option. When submitting the article the author must go through the editorial procedure which is a Double-blind review, this means authors must submit a blinded manuscript without any author names and affiliations. One interesting piece of information about this journal is that electronic multimedia files such as movies, and animations can also be submitted. This means that the journal allows authors to express their papers in many different ways. This journal does not have a page limit on their submission. This journal has 1 volume with 4 issues annually.

\section{Part 2}
\subsection{KeyNote 1 G.Cormode}

There are at most 2 font sizes which is recommended. The color of some words are difficult to weird. For example on page 5 there are highlighted words in neon green. On page 6 of the slide animations are used to give examples. However the slide is scattered and unclear about what is trying to be said. In general the slides are text heavy which means most of the slides have 10 or more lines of text. After slide 19 the slides stop numbering the pages. Also there are many instance where acronyms are not defined. Overall this slide deserves a D grade.

\subsection{KeyNote 2 W. Lehner}

These slide are much more sophisticated than the previous one. It incorporates visuals well. The animations have a clear flow with arrows and labels describing the images. Since the lines of text are few on each slide it allows the audience to narrow down their focus on specific topics. Essentially the message being conveyed is clear and organized. However this presentation does lack a roadmap. Since talks are inherently linear the audience must know the order of events in the talk. Also there is a lack of summary slides, the slides go on and on without reviewing the topics covered. Perhaps if the speaker uses backward and forward references: "As dicussed earlier,..." or "Later on, I will show that..." are legitimate connections to past or future topics. I'd like to give this slide a B- grade because there are ways for the speaker to make up for the lack of summaries and reviews. Also the audience can read the slides since the color of the font is clear and the text is minimal.


\subsection{KeyNote 3 R. Pagh}

This is clearly the best presentation slides out of all the options. The black background and the white font is suitable for the audience has the fonts are also in large size. The slides start off with some tables and and animations. This is good for speaking points and getting the audience interested and engaged in the presentation. By slide 5 there are a list of topics that will be discussed. Immediately after there is an outline which allows the audience to have a mental road map about what discussion points will be addressed. Since the slides are minimal they allow the speaker to have more power in speaking. If the speaker is charismatic and
good with communication then the entire presentation will be enjoyable. For these reasons I give the slides an A
\end{document}
